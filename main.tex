\documentclass[12pt]{article}
\usepackage[spanish]{babel}
\usepackage{amsmath}
\usepackage{graphicx}

\begin{document}

\begin{center}
\bf{\sc\Huge Titulo ensayo}\\
\end{center}
\vspace{120pt}
\begin{center}
\bf{\sc\Huge Manuel Alejandro Lopez Loaiza }\\
\end{center}
\vspace{200pt}
\begin{center}
\bf{\sc\Huge Universidad de Antioquia}
\end{center}
\begin{center}
\bf{\sc\Huge facultad de ingeniería}\\
\end{center}\
\begin{center}
\bf{\sc\Huge ingeniería electrónica}
\end{center}
\begin{center}
\bf{\sc\Huge Medellin}\\
\end{center}\
\begin{center}
\bf{\sc\Huge 2020}\\
\end{center}\



\newpage



\begin{center}

\bf{\sc\Huge Todo esta en las matematicas }\\
\end{center}
\vspace{25PT}
\section{ 3 IDIOTAS}
\large
El comienzo de un nuevo siglo trae consigo una serie de sucesos y hechos que ponen en jaque y cuestiona-miento varios enunciados que se su ponen son validos e irrefutables. A principio del siglo XX se produjo en el campo de las matemáticas un acontecimiento que tiene por nombre “La crisis de los fundamentos”, algo controversial para su época pues se creía que en el mundo de las ciencias exactas todo era axiomático.
Fue en el año 1900 durante el Congreso Internacional de Matemáticos celebrado en Paris que el alemán David Hilbert presento una lista que contenía 28 problemas matemáticos; de los cuales hoy en día la gran mayoría han sido resueltos y solo 7 se mantienen como incógnitas. Es entonces que a finales de los años 20 el joven matemático Kurt Godel tratara de dar una respuesta al problema presentado por Hilbert, específicamente el segundo de la lista, el que busca demostrar que los axiomas de la aritmética tienen consistencia lógica, es decir que la aritmética es un proceso formal que no tiene ningún tipo de contradicción.



\vspace{10PT}
Godel propuso entonces el Teorema de Incompletitud, en el cual afirma que para cualquier conjunto de axiomas es posible hacer enunciados, que a partir de esos axiomas no puede demostrarse ni que son así, ni que no son así. En palabras mas sencillas se tiene claro que existen muchas expresiones matemáticas que son completamente demostrables aritméticamente, sin embargo hay expresiones 



 








\end{document}
